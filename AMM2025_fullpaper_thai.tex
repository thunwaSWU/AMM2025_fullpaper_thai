%====================================================
% MUST COMPILE WITH XeLaTeX
% Put font files in the same folder as the .tex file
%====================================================
% Proceedings Template (THA)
% The 29th Annual Meeting in Mathematics (AMM 2025)
% Srinakharinwirot University
%====================================================

%!TEX TS-program = xelatex
%!TEX encoding = UTF-8 Unicode

\documentclass[12pt, a4paper, twoside]{article}

\usepackage{geometry,graphicx}
\usepackage{amssymb, amsmath, amsthm}
\usepackage{fancyhdr}
\usepackage{tikz}
%\usepackage[hang,flushmargin]{footmisc}
\geometry{top=5cm, bottom=2.5cm, left=2.5cm, right=2.5cm, headheight=3cm}
\usepackage[T1]{fontenc}
\usepackage{indentfirst}

%%%References
\usepackage{etoolbox}
\apptocmd{\thebibliography}{\setlength{\itemsep}{0pt}}{}{}
\usepackage{cite}
\usepackage{enumitem}

%%
\newcommand\blfootnote[1]{\let\thefootnote\relax\footnotetext{\ignorespaces #1}}
\addtolength{\footnotesep}{1pt}

\AddToHook{begindocument/end}{\renewcommand{\thefootnote}{\fnsymbol{footnote}}}

%%%%%%%%%%%%%%%%%%%%%%%%%%
% SETUP FOR THAI LANGUAGE
%%%%%%%%%%%%%%%%%%%%%%%%%%

\usepackage{polyglossia}
\usepackage{fontspec}
\usepackage{xltxtra}
\usepackage{xunicode}
\XeTeXlinebreaklocale "th"
\XeTeXlinebreakskip = 0pt plus 0pt 
\defaultfontfeatures{Mapping=tex-text} 

\newfontfamily\thaifont{THSarabunNew.ttf}[
    Path,
    BoldFont= THSarabunNew Bold.ttf,
    ItalicFont=THSarabunNew Italic.ttf,
    BoldItalicFont=THSarabunNew BoldItalic.ttf,
    Scale = 1.35]
\newfontfamily\thaifonttt{Sarabun-Regular.ttf}[
    Path,
    Script = Thai,
    Scale = 0.95]
\setmainlanguage{thai}
\renewcommand{\baselinestretch}{1.3}

\usepackage[Latin,Thai]{ucharclasses}
\setTransitionsForLatin
  {\begingroup\rmfamilylatin}
  {\endgroup}

%%%%%%%%%%%%%%%%%%%%%%%%%%

\newenvironment{AMM-abstractTH}[4][]{
  \begin{center}
    { \renewcommand\textsuperscript[1]{}\par}
    {{\Large\bfseries #2}\par}
    \medskip
    {\large #3\par}
    \bigskip
    {\small #4\par}
    \bigskip\bigskip
    {{\large\bfseries บทคัดย่อ}\par}
  \end{center}
}{ 
    \bigskip
    \hrule
    \bigskip
}

\renewcommand{\headrulewidth}{0pt}

\newcommand{\mykeywordsTH}[1]{%
    \noindent \textbf{Keywords:} #1 \par
}
\newcommand{\myMSC}[1]{
    \noindent \textbf{2020 MSC:} #1 \par
}

\pagestyle{fancy}
\fancyhead[L]{\footnotesize
The 29\textsuperscript{th} Annual Meeting in Mathematics (AMM 2025)\\
Department of Mathematics, Faculty of Science,\\
Srinakharinwirot University,
Bangkok, Thailand}
\fancyhead[R]{\includegraphics[width=2.3cm]{amm2025_logo_small_color.png}}


%%%%%%% Use AMSLaTeX Theorem Style %%%%%%%%%%%%%%%%%%%%%%%%%%%%%%%%%%%%%%%%%
\theoremstyle{plain}
\newtheorem{theorem}{ทฤษฎีบท}[section]
\newtheorem{lemma}[theorem]{บทตั้ง}
\newtheorem{proposition}[theorem]{ประพจน์}
\newtheorem{conjecture}[theorem]{ข้อความคาดการณ์}
\newtheorem{corollary}[theorem]{บทแทรก}
\theoremstyle{definition}
\newtheorem{definition}[theorem]{บทนิยาม}
\newtheorem{example}[theorem]{ตัวอย่าง}
\newtheorem{question}[theorem]{คำถาม}
\newtheorem{problem}[theorem]{ข้อปัญหา}
\theoremstyle{remark}
\newtheorem{remark}[theorem]{หมายเหตุ}
\renewcommand*{\proofname}{พิสูจน์}

\numberwithin{equation}{section}
\renewcommand{\figurename}{ภาพที่}
\renewcommand{\tablename}{ตารางที่}

%\renewcommand{\abstractname}{\large \bf บทคัดย่อ}
%\usepackage{setspace}  %% double spacing - for thai version
%\usepackage{showframe}


%%%%%%%%%%%%%%% DO NOT make any changes above %%%%%%%%%%%%%%%%%%%%%%%%%%%%%
%
%%%%%%%%%%%%%%%%%%%%%%%%%%%% PLEASE CUSTOMIZE  BELOW  %%%%%%%%%%%%%%%%%%%%%%%%%
%
%------------  Insert any required packages and definitions here --------------

%\usepackage{showframe}
% \usepackage{xxx}

\newcommand{\myvec}[1]{\mathbf{#1}}
%%%%%%

%============   END OF CUSTOMIZATION  ===================

\usepackage[hidelinks, hyperfootnotes=false]{hyperref}
\hypersetup{
    colorlinks=true, 
    linkcolor=blue,
    citecolor=blue,
    filecolor=blue,
    urlcolor=blue,
}

\begin{document}

\setcounter{section}{0}

%%%%%%%%%%%%%%%%%%%%%  START YOUR DOCUMENT HERE %%%%%%%%%%%%%%%%%%%%%%%%

%%%%%%%%%%%%%%%%%% ABSTRACT %%%%%%%%%%%%%%%%%%%%
% Use the following command to write your abstract:
%
% \begin{AMM-abstractTH}[]
% {Title}
% {Authors (use \textsuperscript as institution markers)}
% {Institutions (use \textsuperscript as institution markers)}
% Abstract text
% \end{AMM-abstractTH}
%
%%%%%%%%%%%%%%%%%%%%%%%%%%%%%%%%%%%%%%%%%%%%%%%%%%%

\begin{AMM-abstractTH}[]
{ชื่อเรื่อง} %TITLE
{ชื่อผู้แต่งคนแรก\textsuperscript{1,}\footnote{ผู้นำเสนอ (name@email.com)}, ชื่อผู้แต่งคนที่สอง\textsuperscript{1}
และ ชื่อผู้แต่งคนที่สาม\textsuperscript{2,}\footnote{(ตัวอย่าง) ได้รับทุนสนับสนุนจาก...}} %AUTHORS
{\textsuperscript{1}ภาควิชาคณิตศาสตร์ คณะวิทยาศาสตร์
มหาวิทยาลัยศรีนครินทรวิโรฒ 10110\\ \smallskip
\textsuperscript{2}ภาควิชาคณิตศาสตร์ สถิติและคอมพิวเตอร์ คณะวิทยาศาสตร์ \\ มหาวิทยาลัยอุบลราชธานี 34190} %AFFILIATIONS

%YOUR ABSTRACT GOES HERE
บทคัดย่อภาษาไทยไม่ควรเกิน 250 คำ 
โดยให้ใช้แบบและขนาดอักษรของข้อความต่าง ๆ ดังในไฟล์นี้ 
และไม่มีการอ้างอิงในบทคัดย่อ แนะนำให้ผู้เขียนใช้ลำดับการเขียนบทคัดย่อดังนี้  

1) ความเป็นมาและความสำคัญของปัญหา: วางคำถามในบริบทกว้าง ๆ และเน้นวัตถุประสงค์ของการวิจัย 

2) วิธีการดำเนินการวิจัย: อธิบายวิธีการหลักที่ใช้ในงานวิจัยสั้นๆ 

3) ผลการวิจัย: สรุปผลการค้นที่พบหลัก ๆ ของบทความ และ 

4) สรุปผลการวิจัยและอภิปรายผล: ระบุข้อสรุปหลักหรือการตีความ 

บทคัดย่อควรแสดงวัตถุประสงค์ของบทความแต่ต้องไม่มีผลที่ไม่ได้นำเสนอและไม่ควรสรุปเกินจริง
ดูข้อมูลเพิ่มเติมเกี่ยวกับระบบการจำแนกวิชาคณิตศาสตร์ (2020 MSC) ได้ที่ \\ \url{https://mathscinet.ams.org/msnhtml/msc2020.pdf}
\end{AMM-abstractTH}

%%%%%%%%%%%%%%%%% Keywords and MSC %%%%%%%%%%%%%%%%%%%%%%

\mykeywordsTH{คำสำคัญ 1, คำสำคัญ 2, คำสำคัญ 3} %Please include 3-5 keywords here. Use comma to separate items in the list.
\smallskip
\myMSC{nnXxx, nnXxx, nnXxx} %Please include MSC here. The first item is the primary MSC. Use comma to separate items in the list.


\section{บทนำ}\label{yourname:intro}
บทความนี้ใช้ฟอนต์ THSarabunNew และ Sarabun ให้วางไฟล์ฟอนต์ทั้ง 5 ไฟล์ไว้ในโฟลเดอร์เดียวกันกับไฟล์ .tex เสนอ ทำการประมวลผลด้วย XeLaTeX (การตัดคำภาษาไทยใน XeLaTeX อาจจะมีความไม่สมบูรณ์ ต้องมีการตรวจทานทุกครั้ง) ใช้การแบ่ง section ด้วยคำสั่งมาตรฐานของ \LaTeX\
 
บทนำควรเป็นการปูพื้นฐานความรู้ที่เกี่ยวข้องกับหัวข้อวิจัย อธิบายถึงความสำคัญและความจำเป็นของการศึกษา รวมถึงระบุวัตถุประสงค์ของงานอย่างชัดเจน ควรครอบคลุมการทบทวนวรรณกรรมที่เกี่ยวข้อง ชี้ให้เห็นถึงช่องว่างทางความรู้หรือประเด็นที่ยังไม่ได้รับการแก้ไข และสรุปคำถามหรือสมมติฐานที่เป็นหัวใจสำคัญของการศึกษา เพื่อสร้างความชัดเจนและเน้นย้ำถึงความสำคัญของงานวิจัยนี้ต่อวงการวิชาการ
 
\section{ความรู้พื้นฐาน}\label{yourname:prelim}

โปรดใช้ระบบของ \LaTeX\ ในการอ้างอิง \cite{yourname:book, yourname:bookthai, yourname:bookchapter} และตรวจสอบให้แน่ใจว่าเอกสารทุกฉบับในบรรณานุกรมนั้นได้รับการอ้างถึงในเนื้อหาด้วยเช่นกัน ให้ใช้รูปแบบการอ้างอิง Elsevier's standard numbered style ทั้งการอ้างอิงภายในบทความและบรรณานุกรม เรียงลำดับเอกสารอ้างอิงตามลำดับที่ปรากฎในบทความ 
ดูตัวอย่างได้ใน source file และดูวิธีการอ้างอิงเพิ่มเติมได้ที่ \url{https://booksite.elsevier.com/9780081019375/content/Elsevier\%20Standard\%20Reference\%20Styles.pdf} ดูวิธีการย่อชื่อวารสารได้ที่ \\ \url{https://mathscinet.ams.org/msnhtml/serials.pdf}

\subsection{หัวข้อย่อย}\label{yourname:intro_I}
พิมพ์ข้อความตรงนี้ เช่นสมการ $y'+4y^2=0$ หรือให้สมการแสดงตรงกลางบรรทัด
\[
	\vec{a}\times\vec{b} = \vec{c}+\sum_{i=1}^n C_i 
\]
ลำดับตัวเลขของสมการจะปรากฎด้านขวาของสมการโดยระบุเป็นหมายเลขของหัวข้อ (section) ตามด้วยลำดับหมายเลขสมการ ดังตัวอย่างต่อไปนี้
\begin{equation}\label{yourname:eq15}
	h =T \left ( \sum_{i=1}^n x_i \otimes y_i \right ).
\end{equation}

label ภายในและ cite ทุกคำสั่ง ควรขึ้นต้นด้วยนามสกุลของผู้แต่งคนแรก ดังนี้
\begin{verbatim}
YourName:ref
\end{verbatim}
ตัวอย่างเช่น ถ้านามสกุลของผู้แต่งคนแรกคือ ``Peters,'' ควรตั้ง label และ cite ว่า
\begin{verbatim}
\label{peters:eq15}
\end{verbatim}

\subsubsection{ห้วข้อย่อยลงไปอีก}
ใช้คำสั่ง \LaTeX สำหรับการอ้างอิงภายใน เช่น ดูตอนที่ \ref{yourname:intro_I} และดูสมการ (\ref{yourname:eq15}) และใช้คำสั่งของ package amsthm สำหรับนิยาม บทตั้ง ทฤษฎีบท ฯลฯ  

\begin{definition}
ให้ $A \subseteq \mathbb{R}^n$ เป็นเซตนูน  เรียกจุด $x \in A$ ว่า\emph{จุดสุดขีด} ถ้า $\dots$ ... 
\end{definition}

\begin{theorem}
	ข้อความทฤษฎีบทเริ่มพิมพ์ตรงนี้
\end{theorem}
\begin{proof}
   บทพิสูจน์ทิ้งไว้ให้เป็นแบบฝึกหัดของผู้อ่าน
\end{proof}

\begin{lemma}
	พิมพ์บทตั้งตรงนี้
\end{lemma}

\section{ผลการศึกษา}
ใช้ floats สำหรับรูปภาพและตาราง โดยวางไฟล์ภาพไว้ในโฟลเดอร์เดียวกันกับไฟล์ .tex\footnote{This is a regular footnote.}
     
\begin{figure}[h]
\centering
\includegraphics[scale=0.1]{amm2025_logo_small_color.png}
\caption{AMM 2025 Srinakharinwirot University 21-23 May 2025}
\label{yourname:ammlogo}
\end{figure}

\begin{table}[h]
\caption{เจ้าภาพจัดงาน AMM}
\begin{center}
\begin{tabular}{clc}  \hline
 AMM & เจ้าภาพ & ปี\\ \hline\hline
 $26^{th}$ & SUT &$2022$ \\ 
$27^{th}$ & KU  &$2023$ \\ 
$28^{th}$ & UBU &$2024$ \\ 
$29^{th}$ & SWU &$2025$ \\ 	 
\hline
\end{tabular}
\label{yourname:tableofamm}
\end{center}
\end{table}

\section{สรุปผลและอภิปรายผล}
ส่วนอภิปรายควรเป็นการตีความผลการศึกษา เชื่อมโยงข้อมูลที่ได้กับงานวิจัยที่มีอยู่และเน้นย้ำถึงความสำคัญของผลการวิจัย นอกจากนี้ควรกล่าวถึงข้อจำกัดของการศึกษาอย่างตรงไปตรงมา และเสนอแนวทางสำหรับการวิจัยในอนาคตโดยเน้นถึงผลกระทบที่กว้างขึ้นของผลที่ได้รับ

\bigskip\noindent
\textbf{กิตติกรรมประกาศ} (optional) ผู้แต่งขอขอบคุณผู้ทรงคุณวุฒิทุกท่านที่ได้ให้ข้อคิดเห็นและข้อเสนอแนะต่าง ๆ เพื่อปรับปรุงบทความวิจัยนี้ 

\renewcommand{\refname}{เอกสารอ้างอิง}

\begin{thebibliography}{99}
%Please type your references directly into the source file.
%Please order your bibliography items in the order they appear in the text.

%%%Book
\bibitem{yourname:book}
D. Gopal, P. Kumam, M. Abbas, Background and Recent Developments of Metric Fixed Point Theory, Taylor \& Francis Group LLC, New York, 2017.

\bibitem{yourname:bookthai}
อัจฉรา หาญชูวงศ์, ทฤษฎีจำนวน, โรงพิมพ์แห่งจุฬาลงกรณ์มหาวิทยาลัย, กรุงเทพมหานคร, 2542.

%%%Book Chapter
\bibitem{yourname:bookchapter}
A. Dorko, What do we know about student learning from online mathematics homework?, in: J.P. Howard II, J.F. Beyes (Eds.), Teaching and Learning Mathematics Online, Boca Raton, C\&H/CRC Press, pp. 17--42.

\bibitem{yourname:bookchapterthai}
วรวรรณ จิตต์ธรรม, โรคหัวใจพิการแต่กำเนิดที่พบได้บ่อย, ใน: วรวรรณ จิตต์ธรรม, จิรนันท์ วีรกุล, ญาศินี อภิรักษ์นภานนท์, ชุติมา เผือกสามัญ (บ.ก.), กุมารเวชศาสตร์ในเวชปฏิบัติ, สํานักพิมพ์มหาวิทยาลัยนเรศวร, พิษณุโลก, น. 243--280.

%%% Journal article
\bibitem{yourname:article} T. Theerakarn, On the center of surface area of the boundary of a star-shaped region, College Math. J. 54 (3) (2023) 326--336.

%Multiple authors : less than 7
\bibitem{yourname:multiple} S. Isariyapalakul, W. Pho-on, V. Khemmani, The true twin classes-based investigation for connected local dimensions of connected graphs, AIMS Math. 9 (4) (2024) 9435--9446.

%Multiple authors : 7 or more
\bibitem{yourname:seven} S. Weikert, D. Freyer, M. Weih, N. Isaev, C. Busch, J. Schultze, et al., Rapid Ca\textsuperscript{2+}
-dependent NO-production from central nervous system cells in culture measured by NO-nitrite/ozone chemoluminescence, Brain
Res. 748 (1997) 1--11.

%Article in other language
\bibitem{yourname:lang} 
คุณัชญ์ ศิริสมบูรณ์เวช, ชัญญา เล็กเจริญศรี, ณหทัย ฤกษ์ฤทัยรัตน์, ศิริสุดา อินสกุล, เกมลูกเต๋าฮอกกับกติกาเพิ่มเติม, วารสารคณิตศาสตร์โดยสมาคมคณิตศาสตร์แห่งประเทศไทย ในพระบรมราชูปถัมภ์. 68 (709) (2566) 1--18.

%%% Article in Conference Proceedings
\bibitem{yourname:proceedings}
T.E. Chaddock, Gastric emptying of a nutritionally balanced liquid diet, in: E.E. Daniel (Ed.),
Proceedings of the Fourth International Symposium on Gastrointestinal Motility, ISGM4, 4--8 September 1973, Seattle,
WA, Mitchell Press, Vancouver, British Columbia, Canada, 1974, pp. 83--92.

%%% Website
\bibitem{yourname:website} Cancer Research UK, Cancer statistics reports for the UK.
 <\url{http://www.cancerresearchuk.org/aboutcancer/statistics/cancerstatsreport/}>, 2003 (accessed 13.03.03).
\bibitem{yourname:websitethai} ภาควิชาคณิตศาสตร์ คณะวิทยาศาสตร์ มหาวิทยาลัยศรีนครินทรวิโรฒ, ประวัติการประชุมวิชาการทางคณิตศาสตร์ระดับชาติ.
 <\url{http://www.amm2025.com}>, 2567 (สืบค้น 16 ม.ค. 68).
 
%%% Translated Book
\bibitem{yourname:translated} A.R. Luria, The Mind of a Mnemonist (L. Solotarof, Trans.), Avon Books, New York, 1969 (Original
work published 1965).

\bibitem{yourname:translated_thai} ดี.ดี. ไฮเซอรอตต์, มนุษยสัมพันธ์: กลยุทธ์การสื่อสารเพื่อความสำเร็จของธุรกิจ (วัฒนา พัฒนพงศ์,
ผู้แปล), ต้นอ้อ, กรุงเทพมหานคร, 2536 (ต้นฉบับพิมพ์ปี ค.ศ. 1990).

%%% Thesis
\bibitem{yourname:thesis} T. Theerakarn, Locally volume collapsed 4-manifolds with respect to a lower sectional curvature bound, (Doctoral dissertation), UC Berkeley, 2018 \url{https://escholarship.org/uc/item/0x37d5vr}.
\bibitem{yourname:thesisthai} อรทัย จุลสุวรรณรักษ์, การค้ามนุษย์ในประเทศไทย, (วิทยานิพนธ์ปริญญามหาบัณฑิต), มหาวิทยาลัยธรรมศาสตร์, 2550.

\end{thebibliography}

\end{document}
